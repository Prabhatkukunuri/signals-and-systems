% \iffalse
\let\negmedspace\undefined
\let\negthickspace\undefined
\documentclass[journal,12pt,twocolumn]{IEEEtran}
\usepackage{cite}
\usepackage{amsmath,amssymb,amsfonts,amsthm}
\usepackage{algorithmic}
\usepackage{graphicx}
\usepackage{textcomp}
\usepackage{xcolor}
\usepackage{txfonts}
\usepackage{listings}
\usepackage{enumitem}
\usepackage{mathtools}
\usepackage{gensymb}
\usepackage{comment}
\usepackage[breaklinks=true]{hyperref}
\usepackage{tkz-euclide} 
\usepackage{listings}
\usepackage{gvv}                                        
\def\inputGnumericTable{}                                 
\usepackage[latin1]{inputenc}                                
\usepackage{color}                                            
\usepackage{array}                                            
\usepackage{longtable}                                       
\usepackage{calc}                                             
\usepackage{multirow}                                         
\usepackage{hhline}                                           
\usepackage{ifthen}                                           
\usepackage{lscape}

\newtheorem{theorem}{Theorem}[section]
\newtheorem{problem}{Problem}
\newtheorem{proposition}{Proposition}[section]
\newtheorem{lemma}{Lemma}[section]
\newtheorem{corollary}[theorem]{Corollary}
\newtheorem{example}{Example}[section]
\newtheorem{definition}[problem]{Definition}
\newcommand{\BEQA}{\begin{eqnarray}}
\newcommand{\EEQA}{\end{eqnarray}}
\newcommand{\define}{\stackrel{\triangle}{=}}
\theoremstyle{remark}
\newtheorem{rem}{Remark}
\begin{document}
\parindent 0px

\bibliographystyle{IEEEtran}
\vspace{3cm}

\title{Assignment\\[1ex]11.9.2 - 11}
\author{EE23BTECH11034 - Prabhat Kukunuri$^{}$% <-this % stops a space
}
\maketitle
\newpage
\bigskip

\renewcommand{\thefigure}{\theenumi}
\renewcommand{\thetable}{\theenumi}
\section*{Question}
Sum of the first p, q and r terms of an A.P. are a, b and c, respectively.

Prove that $\dfrac{a}{p}\brak{q-r}+\dfrac{b}{q}\brak{r-p}+\dfrac{c}{r}\brak{p-q}=0$
\section*{Solution}
\begin{table}[h]
    \centering
    \begin{tabular}{|p{2.5cm}|p{2.5cm}|p{2.55cm}|}
\hline
Symbol&Value&Description\\ \hline
$x(n)$&$(x(0)+nd)\times u(n)$&$n^{th}$ term of an A.P\\ \hline
$x(0)$&$x(0)$&$1^{st}$ term of the A.P\\ \hline
$d$&$d$&Common difference\\ \hline
$u(n)$&unit step function&$u(n)=0$ \brak {n<0}   $u(n)=1$ \brak {n\geq0}\\ \hline
\end{tabular}
    \caption{Variable description}
    \label{tab:11.9.2.11.1}
\end{table}
\begin{align}
	x \brak{n} &\system{Z} X \brak{z} \\
    X \brak{z} &= \sum_{n=-\infty}^{\infty} x \brak{n}   z^{-n}\\
    X \brak{z} & = \sum_{n=-\infty}^{\infty}\brak{x\brak{0}+nd}u\brak{n}z^{-n}\\
    u \brak{n} &\system{Z} U \brak{z} = \dfrac{1}{1-z^{-1}}, |z|>1\\
    X \brak{z} & = \dfrac{x\brak{0}}{1-z^{-1}} + \dfrac{dz^{-1}}{\brak{1-z^{-1}}^{2}}\\
    y \brak{n} &\system{Z} Y \brak{z}\\
    Y\brak{z}&=\sum_{n=-\infty}^{\infty}y\brak{n}z^{-n}
\end{align}
\begin{align}
    y\brak{n}&=x\brak{n}\ast u\brak{n}\\
    Y\brak{z}&=X\brak{z}U\brak{z}\\
    Y\brak{z}&=\brak{\dfrac{x\brak{0}}{1-z^{-1}} + \dfrac{dz^{-1}}{\brak{1-z^{-1}}^{2}}}\brak{\dfrac{1}{1-z^{-1}}}\\
    &n^2u\brak{n}\system{Z} \dfrac{z^{-1}+z^{-2}}{\brak{1-z^-1}^3}
 \end{align}
 By performing inverse Z transform on Y\brak{z}
\begin{align}
    y\brak{n}=x\brak{0}\brak{n+1}u\brak{n}+d\brak{\dfrac{{n\brak{n+1}}}{2}}u\brak{n}\\
    y\brak{n}=\dfrac{n+1}{2}\brak{2x\brak{0}+nd}\\
    a=\dfrac{p}{2}\brak{2x\brak{0}+\brak{p-1}d}\\
    b=\dfrac{q}{2}\brak{2x\brak{0}+\brak{q-1}d}\\
    c=\dfrac{r}{2}\brak{2x\brak{0}+\brak{r-1}d}
\end{align}
Back substituting values into the term $\dfrac{a}{p}\brak{q-r}$ it can be rewritten as $\dfrac{p}{2} \times \dfrac{1}{p}\brak{q-r}\brak{2x+\brak{p-1}d}$

On further simplification it can be rewritten as 
\begin{align}
    \dfrac{\brak{q-r}}{2}\brak{2x\brak{0}-d+pd}
\end{align}
Assuming $2x\brak{0}-d$ as a constant $k$
\begin{align}
    \dfrac{a}{p}\brak{q-r} = \dfrac{\brak{q-r}}{2}\brak{k+pd}\\
    \dfrac{\brak{q-r}}{2}\brak{k+pd} = \dfrac{kq+pqd-rk-prd}{2}\label{eq:7}\\
    \dfrac{\brak{r-p}}{2}\brak{k+qd} = \dfrac{kr+qrd-pk-pqd}{2}\label{eq:8}\\
    \dfrac{\brak{p-q}}{2}\brak{k+rd} = \dfrac{kp+prd-qk-qrd}{2}\label{eq:9}
\end{align}
Upon on addition of $ \eqref{eq:7}$,$ \eqref{eq:8}$ and$ \eqref{eq:9}$ the total sum adds up to 0.
\end{document}