% \iffalse
\let\negmedspace\undefined
\let\negthickspace\undefined
\documentclass[journal,12pt,twocolumn]{IEEEtran}
\usepackage{cite}
\usepackage{amsmath,amssymb,amsfonts,amsthm}
\usepackage{algorithmic}
\usepackage{graphicx}
\usepackage{textcomp}
\usepackage{xcolor}
\usepackage{txfonts}
\usepackage{listings}
\usepackage{enumitem}
\usepackage{mathtools}
\usepackage{gensymb}
\usepackage{comment}
\usepackage[breaklinks=true]{hyperref}
\usepackage{tkz-euclide} 
\usepackage{listings}
\usepackage{gvv}                                        
\def\inputGnumericTable{}                                 
\usepackage[latin1]{inputenc}                                
\usepackage{color}                                            
\usepackage{array}                                            
\usepackage{longtable}                                       
\usepackage{calc}                                             
\usepackage{multirow}                                         
\usepackage{hhline}                                           
\usepackage{ifthen}                                           
\usepackage{lscape}

\newtheorem{theorem}{Theorem}[section]
\newtheorem{problem}{Problem}
\newtheorem{proposition}{Proposition}[section]
\newtheorem{lemma}{Lemma}[section]
\newtheorem{corollary}[theorem]{Corollary}
\newtheorem{example}{Example}[section]
\newtheorem{definition}[problem]{Definition}
\newcommand{\BEQA}{\begin{eqnarray}}
\newcommand{\EEQA}{\end{eqnarray}}
\newcommand{\define}{\stackrel{\triangle}{=}}
\theoremstyle{remark}
\newtheorem{rem}{Remark}
\begin{document}
\parindent 0px

\bibliographystyle{IEEEtran}
\vspace{3cm}

\title{Assignment\\[1ex]11.9.2 - 11}
\author{EE23BTECH11034 - Prabhat Kukunuri$^{}$% <-this % stops a space
}
\maketitle
\newpage
\bigskip

\renewcommand{\thefigure}{\theenumi}
\renewcommand{\thetable}{\theenumi}
\section*{Question}
Sum of the first p, q and r terms of an A.P. are a, b and c, respectively.

Prove that $\dfrac{a}{p}(q-r)+\dfrac{b}{q}(r-p)+\dfrac{c}{r}(p-q)=0$
\section*{Solution}
\begin{table}[h]
    \centering
    \begin{tabular}{|p{2.5cm}|p{2.5cm}|p{2.5cm}|}
    \hline
   Symbol&Value&Description\\ \hline
   $x(n)$&{$\dfrac{n}{2}(2a+(n-1)d)$}&Sum of n terms of an A.P\\ \hline
   $n$&$p,q,r$&$n^{th}$ term of the sequence\\ \hline
   $a$&$x(0)$&first term of the sequence\\ \hline
   $d$&\tiny$x(n+2)-2x(n+1)+x(n)$&common difference\\ \hline

    \end{tabular}
    \caption{Variable description}
    \label{tab:11.9.2.11}
\end{table}
\begin{align}
    a=\dfrac{p}{2}(2x(0)+(p-1)d)\\
    b=\dfrac{q}{2}(2x(0)+(q-1)d)\\
    c=\dfrac{r}{2}(2x(0)+(r-1)d)
\end{align}
Back substituting values into the term $\dfrac{a}{p}(q-r)$ it can be rewritten as $\dfrac{p}{2} \times \dfrac{1}{p}(q-r)(2x+(p-1)d)$

On further simplification it can be rewritten as 
\begin{align}
    \dfrac{(q-r)}{2}(2x(0)-d+pd)
\end{align}
Assuming $2x(0)-d$ as a constant $k$
\begin{align}
    \dfrac{a}{p}(q-r) = \dfrac{(q-r)}{2}(k+pd)\\
    \dfrac{(q-r)}{2}(k+pd) = \dfrac{kq+pqd-rk-prd}{2}\label{eq:7}\\
    \dfrac{(r-p)}{2}(k+qd) = \dfrac{kr+qrd-pk-pqd}{2}\label{eq:8}\\
    \dfrac{(p-q)}{2}(k+rd) = \dfrac{kp+prd-qk-qrd}{2}\label{eq:9}
\end{align}
Upon on addition of \eqref{eq:7}, \eqref{eq:8} and \eqref{eq:9} the total sum adds up to 0.
\end{document}