% \iffalse
\let\negmedspace\undefined
\let\negthickspace\undefined
\documentclass[journal,12pt,twocolumn]{IEEEtran}
\usepackage{cite}
\usepackage{amsmath,amssymb,amsfonts,amsthm}
\usepackage{algorithmic}
\usepackage{graphicx}
\usepackage{textcomp}
\usepackage{xcolor}
\usepackage{txfonts}
\usepackage{listings}
\usepackage{enumitem}
\usepackage{mathtools}
\usepackage{gensymb}
\usepackage{comment}
\usepackage[breaklinks=true]{hyperref}
\usepackage{tkz-euclide} 
\usepackage{listings}
\usepackage{gvv}                                        
\def\inputGnumericTable{}                                 
\usepackage[latin1]{inputenc}                                
\usepackage{color}                                            
\usepackage{array}                                            
\usepackage{longtable}                                       
\usepackage{calc}                                             
\usepackage{multirow}                                         
\usepackage{hhline}                                           
\usepackage{ifthen}                                           
\usepackage{lscape}
\newtheorem{theorem}{Theorem}[section]
\newtheorem{problem}{Problem}
\newtheorem{proposition}{Proposition}[section]
\newtheorem{lemma}{Lemma}[section]
\newtheorem{corollary}[theorem]{Corollary}
\newtheorem{example}{Example}[section]
\newtheorem{definition}[problem]{Definition}
\newcommand{\BEQA}{\begin{eqnarray}}
\newcommand{\EEQA}{\end{eqnarray}}
\newcommand{\define}{\stackrel{\triangle}{=}}
\theoremstyle{remark}
\newtheorem{rem}{Remark}
\begin{document}
\parindent 0px

\bibliographystyle{IEEEtran}
\vspace{3cm}

\title{Assignment\\[1ex]GATE-EC-39}
\author{EE23BTECH11034 - Prabhat Kukunuri$^{}$% <-this % stops a space
}
\maketitle
\newpage
\bigskip

\renewcommand{\thefigure}{\theenumi}
\renewcommand{\thetable}{\theenumi}
\section{Question}
Consider the circuit shown in the figure with input V(t) in volts.The sinusoidal steady state current I(t) flowing through the circuit is shown graphically(where t is in seconds). The circuit element Z can be\rule{1.5cm}{0.15mm}.
\begin{enumerate}
    \item a capacitor of 1 F
    \item an inductor of 1 H
    \item a capacitor of $\sqrt{3}$ H
    \item an inductor of $\sqrt{3}$ H
\end{enumerate}
\begin{figure}[ht]
    \centering
    \includegraphics[width=\columnwidth]{figs/Figure_1.png}
    \label{fig:GATE.2022.EC.39.1}
\end{figure}
\solution\\
The current through the circuit can be expressed as
\begin{align}
    I(t)=\sin\brak{t-\frac{\pi}{4}}
\end{align}
Since, the voltage seems to be leading the current the circuit element z is an inductor with inductance L.\\
Applying KVL in the circuit,
\begin{align}
    R.I\brak{t}+L\frac{dI\brak{t}}{dt}=sin\brak{t}
\end{align}
Applying Fourier transform to the differential equation,
\begin{align}
    &R.I\brak{s}+sL.I\brak{s}-\frac{1}{s^2+1}=0\\
    &I\brak{s}\brak{R+sL}=\frac{1}{s^2+1}\\
    &\sin\brak{at+b}\system{L}\frac{a\cos\brak{b}+s\sin\brak{b}}{a^{2}+s^{2}}\\
    &\sin\brak{t-\frac{\pi}{4}}\system{L}\frac{1-s}{2\brak{s^2+1}}\\
    &\frac{1-s}{2\brak{s^2+1}}\brak{R+sL}=\frac{1}{s^2+1}
\end{align}
Upon plugging in R=1$\ohm$,
\begin{align}
   L=\frac{1}{s}
\end{align}
Applying inverse Laplace,
\begin{align}
    L=1H
\end{align}

Appendix\\
Laplace transform of $\sin\brak{at+b}$ is as follows,
\begin{align}
    &\sin\brak{at+b}\system{L}\int_{0}^{\infty}\sin\brak{at+b}e^{-st}dt\\
    &\int_{0}^{\infty}\sin\brak{at+b}e^{-st}dt=\cos{b}\int_{0}^{\infty}\sin\brak{at}e^{-st}dt+\sin{b}\int_{0}^{\infty}\cos\brak{at}e^{-st}dt\\
    &\int_{0}^{\infty}\cos\brak{at}e^{-st}dt=\frac{e^{-st}}{a}sin{at}\Bigg|_{0}^{\infty}+\frac{s}{a}\int_{0}^{\infty}\sin\brak{at}e^{-st}dt\\
    &\int_{0}^{\infty}\cos\brak{at}e^{-st}dt=\frac{s}{a}\int_{0}^{\infty}\sin\brak{at}e^{-st}dt\label{eq:GATE.2022.EC.39.2}\\
    &\int_{0}^{\infty}\cos\brak{at}e^{-st}dt=\frac{s}{a}\brak{\frac{-e^{-st}}{a}cos{at}\Bigg|_{0}^{\infty}+\frac{s}{a}\int_{0}^{\infty}\cos\brak{at}e^{-st}dt}\\
    &\int_{0}^{\infty}\cos\brak{at}e^{-st}dt=\frac{s}{a^2}+\frac{s^2}{a^2}\int_{0}^{\infty}\cos\brak{at}e^{-st}dt
\end{align}
\begin{align}
    &\int_{0}^{\infty}\cos\brak{at}e^{-st}dt=\frac{s}{s^2+a^2},s>0
\end{align}
From \eqref{eq:GATE.2022.EC.39.2} we can say,
\begin{align}
    &\int_{0}^{\infty}\sin\brak{at}e^{-st}dt=\frac{a}{s^2+a^2},s>0\\
    &\therefore \sin\brak{at+b}\system{L}\frac{s\sin{b}+a\cos{b}}{s^2+a^2}
\end{align}
\end{document}
